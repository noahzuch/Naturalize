\subsection{External Validity}
Threats to \emph{external validity} are concerned with the possibility, that the results of the evaluation can not be generalized outside of the testing setup scope. The main factor to reason about in this study is the selection of projects used for the evaluation. Allamanis et al.~\cite{naturalize} have already discussed this problem and came to the conclusion, that their careful selection of the testing projects ensures the generalizability of the results. This can also be directly applied to the evaluation made in this paper, as it uses the same test setup.

\subsection{Internal Validity}
Threats to \emph{internal validity} are concerned with the possibility that the relationship between treatment and outcome of an evaluation is not causal, but rather influenced by factors over which the evaluation has no control or does not concretely specify. Some small tweaks to the original source code of Naturalize were performed, which could potentially change the outcome of the tool. But these changes are only to make Naturalize more performant in the tests and do not change any logic. The result of the reevaluation supports this claim. The comparison between the results of the original study and the reevaluation supports this claim.
As the setup of the evaluation between the baseline and the modified version only differs in one specific change, no uncontrolled factors should exist. To make sure, that the evaluation is specified as concretely as possible, the source code of the modifications and evaluation is made publicly available under the following link: \href{A Link between Sites}{\url{https://de.wikipedia.org/wiki/The_Legend_of_Zelda:_A_Link_Between_Worlds}}.

\subsection{Conclusion Validity}
Threats to \emph{conclusion validity} limit the degree to which conclusions about the relationship between variables from the resulting data are correct or reasonable. Different from most other machine learning tools, N-gram models are a deterministic approach so the results can not be influenced by some random variable. Also, the results from the original study are used to bring the observed values into context.